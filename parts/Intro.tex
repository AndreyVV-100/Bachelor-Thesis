\section{Введение}
\label{sec:Intro} \index{Intro}

Нейронные сети в последниее время активно развиваются и находят применение
в большом количестве различных областей, в том числе очень популярными стали
средства для обработки естественной речи. Количество операций
нейронных сетей и их параметров непрерывно увеличивается, поэтому для их
выполнения требуются существенные вычислительные мощности.
Для выполнения современных нейронных сетей разработчики аппаратного обеспечения
развивают специализированные процессорные архитектуры, которые могут быть
использованы в серверных или мобильных решениях. Наиболее известными из них
являются процессор Ascend с архитектурой DaVinci компании Huawei,
NeuroMorphic Processor компании LG, Google TPU, а также российская разработка ---
векторно-матричная архитектура NeuroMatrix.

Задача эффективного использования возможностей архитектуры нейропроцессоров
полностью делегируется компилятору и среде времени выполнения.
Значительная часть существующих компиляторов с открытым исходным кодом, как
правило, ориентированы на генерацию эффективного кода для моделей вычислений,
используемых графическими процессорами или процессорами общего назначения с
векторными расширениями. Одними из наиболее популярных проектов, позволяющих
создавать компиляторы нейронных сетей являются проекты TVM и Halide. Данные
проекты предоставляют примитивы для описания стратегии выполнения операций
линейной алгебры. Однако использование инфраструктуры этих проектов
для генерации эффективого кода специализированных нейропроцессоров может
оказаться затруднительным. На сегодняшний день перспективным проектом
для создания компиляторов является LLVM MLIR
\cite{mlir-doc, mlir-article1, mlir-article2, mlir-article3}.

Инфраструктура MLIR предоставляет широкий набор примитивов для проектирования
собственных промежуточных представлений, трансформирующих преобразований над
ними, а также трансляции между ними. Описание инструкций промежуточного
представления выполняется на декларативном языке, а сами инструкции могут
обладать произвольной семантикой: скалярной, векторной, матричной и другими.

В данной работе исследованы особенности архитектуры DaVinci и представлены
способы генерации эффективого машинного кода с использованием инфраструктуры
MLIR путём трансляции крупноблочных операторов в операторы, соответствующие
целевой архитектуре.

\newpage
