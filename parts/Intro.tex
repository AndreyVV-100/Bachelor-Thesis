\section{Введение}
\label{sec:Intro} \index{Intro}

Нейронные сети в последниее время активно развиваются и находят применение
в большом количестве различных областей, в том числе очень популярными стали
средства для обработки естественной речи. Стоит отметить, что развитие
нейросетей происходит не только за счёт улучшения архитектура и точности
полученных ответов. Например, ведётся разработка способов ускорить исполнение
уже обученных моделей. Для этого разрабатываются процессоры с матричной
архитектурой, которые могут быть использованы в серверных или мобильных
решениях.

Для использования таких процессоров необходим широкий набор утилит, в том
числе и компиляторы. Как известно, основной задачей любого компилятора
является получение наиболее оптимального с точки зрения производительности
машинного кода при сохранении всех свойств исходной программы. Заметим, что в
таких компиляторах помимо традиционных техник оптимизации, таких как удаление
мёртвого кода, распостранения констант, сокращения общих подвыражений и других,
должны применяться другие техники, связанные с математическими свойствами
тензоров и спецификой целевой архитектуры.

В силу описанных выше причин идут активные исследования в области компиляторов
для нейронных сетей, в том числе и нашей лабораторией. В данной работе будет
исследованы особенности целевой архитектуры и представлены способы генерации
эффективого машинного кода.

\newpage
