\subsection{Общая схема компиляции и возможные проблемы}
\label{impl:problem}

В главе FIXME было дано подробное описание особенностей работы процессоров Ascend.
Для написания эффективного компилятора необходимо учесть их и отразить
в создаваемой инфраструктуре. Кратко обозначим эти проблемы и пути их решения:

\begin{enumerate}
    \item Гетерогенность системы. Работа с Ascend является примером гетерогенных
          вычилений. Поэтому компилятор должен генерировать отдельно код выполнения
          для хоста, и отдельно для девайса. Первый должен поддержить граф
          вычислений, аллоцировать память, планировать исполнение раличных
          функций девайса. Код девайса должен быть как можно более оптимальным,
          так как именно он занимает большую часть времени. Отметим, что далее
          в данной работе будет рассматриваться только компиляция кода девайса.
    
    \item Типы данных. Очень часто входные данные имеют тип \texttt{float 32},
          но матричные вычисления на Ascend возможны только с типами
          \texttt{float 16} или \texttt{int 8}. По этой причине данные
          конвертируются в тип для вычислений. После выполнения операции,
          вероятно, необходима обратная конвертация.

    \item Форматы данных. Напомним, матрицы для умножения должны иметь блочный
          формат хранения (\texttt{Zz}, \texttt{Zn} или \texttt{Nz}). Входные
          данные, как правило, не соответствуют ему, поэтому перед умножением
          необходимо выполнять операцию \textit{фрактализации}, т.~е. изменения
          формата хранения. Аналогично предыдущему пункту, после вычислений
          используется \textit{дефрактализация}.

    \item Использование памяти. Размеры внутренних буферов крайне ограничены,
          по этой причине для исполнения одной операции, зачастую, делается
          несколько загрузок из внешней памяти, которые могут повторяться в
          силу свойств операции (например, в случае блочного умножения матриц).
          Количество загрузок, особенно повторных, должно быть минимальным,
          т.~е. должна быть обеспечена пространственная и временная локальность.

    \item Параллелизм и синхронизация. Как было отмечено, у процессоров с
          архитектурой DaVinci есть шесть юнитов, которые могут работать
          параллельно. По этой причине процент задействования каждого юнита
          (т.~е. время работы юнита по отношению к общему времени работы)
          должно быть максимизировано, для этого синхронизация между операциями
          различных юнитов должна присутствовать только в случае наличия
          зависимости по данным или именам, а само количество зависимостей ---
          минимизировано.

    \item Стратегии и расписание выполнения. Стоит отметить, что в силу
          ограниченности памяти большинство операций будут представлены в виде
          цикла: загрузка-обработка-выгрузка. Очевидно, что загрузка может быть
          для разных по форме и размеру данных, а цикл может иметь различный
          порядок обхода. Таким образом, необходим выбор оптимальной стратегии
          (формы данных) и расписания (порядка обхода). Этот пункт включает
          в себя предыдущие два, но является более общим.
\end{enumerate}
