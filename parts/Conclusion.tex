\section{Заключение и дальнейшая работа}
\label{sec:Conclusion} \index{Conclusion}

В рамках данной работы были решены следующие задачи:

\begin{enumerate}
    \item Исследованы архитектуры нейронных сетей BERT и ResNet,
          основными целями для реализации были выбраны умножение матриц и
          векторные операции.
    \item Исследованы различные инфраструктуры для создания и компиляции
          нейронных сетей (например, PyTorch, Tensorflow), и используемые ими
          аппаратные возможности для увеличения производительности.
    \item Разработан набор операторов для целевой архитектуры DaVinci в
          инфраструктуре MLIR (диалекты CCE и HIVM).
    \item Разработан набор тестов для оценки эффективности стратегий трансляции
          операций нейронных сетей.
    \item Реализованы методы генерации целевых инструкций процессора
          для операций умножения матриц, фрактализации, векторных операций.
\end{enumerate}


Корректность трансляции каждой крупноблочной операции из главы
\ref{impl:lowering} была проверена на синтетическом наборе тестов. Результат
работы оператора сравнивался с эталонным результатом, полученным при исполнении
аналогичного кода на языке Python. Корректность работы всего компилятора
экспериментально исследована и подтверждена на реальной нейронной сети BERT,
содержащей 478 операторов.

Но, несмотря на это, планируются дальнейшие исследования.
На данный момент компилятор находится в стадии активной разработки, поэтому
отметим проблемы, которые были выяснены в процессе исследования и будут решены
в будущем:

\begin{enumerate}
    \item Процессоры Ascend содержат несколько ядер, по этой причине разделение
          крупноблочных операций на несколько независимых частей является одним
          из наиболее простых способов увеличить производительность (см.
          главу \ref{subsubsec:Opt}).
    \item Замеры производительности показали, что двойная буферизация (см.
          главу \ref{subsubsec:Opt}) повышают производительность.
    \item Инструменты синхронизации в текущем компиляторе используются
          неоптимально, улучшение алгоритмов синхронизации ведёт к снижению
          времени исполнения.
    \item Методы полиэдральной компиляции (см. главу \ref{subsec:poly})
          могут быть использованы для оптимизации гнёзд циклов. Необходимы
          дополнительные исследования для изучения этих возможностей.
    \item В операциях нейронных сетей многомерные массивы могут иметь динамические
          размеры (т.~е. неизвестные на этапе компиляции), на данный момент
          их поддержка отсутствует.
\end{enumerate}

\newpage
