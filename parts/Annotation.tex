\begin{abstract}

    \begin{center}
        \large{Разработка компилятора нейронных сетей на основе инфраструктуры MLIR для процессора с матричной архитектурой} \\
    \large\textit{Вязовцев Андрей Викторович} \\[1 cm]

    Краткое описание задачи и основных результатов, мотивирующее прочитать весь текст.

    Пункт 5: разделить на два: 
        1. специализированные процессоры для выполнения (inference) нейронных сетей (мб GPU, LG и DaVinci, сюда перенести
           п. 4 и 7, рассказать про методы оптимизации (двойная буферизация)), сосредоточить внимание на юнитах, привести примеры кода.
        2. компиляторы
            а. Tensorflow, XLA, Onnx, Pytorch, стандарты: TOSA, StableHLO, проблематика: модель -> план, трансляция в код
            б. Полиэдральная компиляция, планирование, преобразования циклов. Промышленное внедрение: TVM, Halide, poly, polygeist.
               Рассмотрение этих компиляторов, примеры (такое будет в диалекте transform). Есть возможность создавать свои операции,
               но это не удобно.
            в. Заключение: компиляторы повторяются, большой количество IR, появление LLVM MLIR.

    Объединить п. 6, 8, 9, 10 в один и разделить на подпункты. Пайплайны хоста и асценда. Примеры? Соответствие операторов и
    ассемблерных инструкций. Проблемы: подготовка данных, эффективное использование памяти, параллелизм, синхронизация,
    расписание (стратегия выполнения цикла).

    Заключение, дальнейшая работа.

    \vfill

    \textbf{Abstract} \\[1 cm]

    FIXME: English abstract? 
    \end{center}

\end{abstract}
\newpage