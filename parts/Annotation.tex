\begin{abstract}

\begin{center}
    \large{Разработка компилятора нейронных сетей на основе инфраструктуры MLIR для процессора с матричной архитектурой} \\
\large\textit{Вязовцев Андрей Викторович} \\[1 cm]
\end{center}

Нейронные сети являются хорошим и эффективным способом обработать какие-либо
данные. По этой причине они находят применение в различных сферах жизни и
становятся всё более популярными. Для увеличения производительности нейросетей
используются различные способы, такие как компиляторы и специальные
вычислительные устройства.

В данной работе исследуются различные способы оптимизации нейронных сетей и
описывается реализация компилятора. Цель работы --- разработать компилятор
нейронных сетей для процессоров Ascend, основанных на архитектуре DaVinci, с
использованием инфраструктуры LLVM MLIR и обеспечить генерацию эффективного
машинного кода в нём. В частности, рассматриваются отдельные крупноблочные
операции нейронных сетей и процесс их трансляции в инструкции целевой
архитектуры.

\vfill

\textbf{Abstract} \\[1 cm]

Neural networks are a good and effective way to process any data.
For this reason, they are used in various areas of life and are becoming
increasingly popular. To increase the performance of neural networks, various
methods are used, such as compilers and special computing devices.

This paper examines various methods for optimizing neural networks and describes
the implementation of the compiler. The goal of the work is to develop a neural
network compiler for Ascend processors based on the DaVinci architecture using
the LLVM MLIR infrastructure and ensure the generation of efficient machine code
in it. In particular, individual large-block operations of neural networks and
the process of their translation into instructions of the target architecture
are considered.

\end{abstract}
\newpage