\begin{center}
\textbf{Аннотация} \\[0.3 cm]
Разработка компилятора нейронных сетей на основе инфраструктуры MLIR для процессора с матричной архитектурой \\
\textit{Вязовцев Андрей Викторович} \\[1 cm]
\end{center}

Современные нейронные сети увеличиваются и требуют большее количество
вычислительных ресурсов. По этой причине разрабатываются специализированное
аппаратное обеспечение, например нейронные процессоры, и оптимизирующие
компиляторы для него. В рамках дипловной работы реализован компилятор нейронных
сетей для процессора с архитектурой DaVinci. Компилятор разработан на базе
инфраструктуры LLVM MLIR. В работе рассмотрены
основные особенности архитектуры и исследованы способы их эффективого
использования. Реализованный компилятор содержит три промежуточных уровня
представления и выполняет трансляцию основных операций
нейронных сетей в целевой набор инструкций архитектуры DaVinci с учётом её
особенностей, таких как: неоднородная организация памяти, наличие матричных и векторных
инструкций. Корректность работы компилятора была экспериментально исследована на
синтетическом наборе тестов, включающих в себя отдельные крупноблочные операции,
а также на реальной нейронной сети BERT, содержащей 478 операторов. \\[1 cm]

\newpage

\begin{center}
    \textbf{Abstract} \\[0.3 cm]
    Development of a neural network compiler based on MLIR for a processor with a matrix architecture \\
    \textit{Vyazovtsev Andrey Victorovich} \\[1 cm]
\end{center}

Modern neural networks are growing in size and require more computing resources.
For this reason, specialized hardware, such as neural processors, and optimizing
compilers for it are being developed. As part of the thesis, a neural network
compiler was implemented for a processor with the DaVinci architecture. The
compiler is developed based on the LLVM MLIR infrastructure. The paper examines
the main features of the architecture and explores ways to use them effectively.
The implemented compiler contains three intermediate levels of representation
and translates the main operations of neural networks into the target set of
instructions of the DaVinci architecture, taking into account its features, such
as: heterogeneous memory organization, the presence of matrix and vector
instructions. The correctness of the compiler's operation was experimentally
studied on a synthetic set of tests, including individual large-block
operations, as well as on a real BERT neural network containing 478 operators.

\newpage