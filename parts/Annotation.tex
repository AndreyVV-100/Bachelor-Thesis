\begin{center}
\textbf{Аннотация} \\[0.3 cm]
Разработка компилятора нейронных сетей на основе инфраструктуры MLIR для процессора с матричной архитектурой \\
\textit{Вязовцев Андрей Викторович} \\[1 cm]
\end{center}

Современные нейронные сети увеличиваются и требуют большее количество
вычислительных ресурсов. По этой причине разрабатываются специализированное
аппаратное обеспечение, например нейронные процессоры, и оптимизирующие
компиляторы для него. В рамках дипловной работы реализован компилятор нейронных
сетей для процессора с архитектурой DaVinci. Компилятор разработан на базе
инфраструктуры LLVM MLIR. В работе рассмотрены
основные особенности архитектуры и исследованы способы их эффективого
использования. Реализованный компилятор содержит три промежуточных уровня
представления и выполняет трансляцию основных операций
нейронных сетей в целевой набор инструкций архитектуры DaVinci с учётом её
особенностей, таких как: неоднородная организация памяти, наличие матричных и векторных
инструкций. Корректность работы компилятора была экспериментально исследована на
синтетическом наборе тестов, включающих в себя отдельные крупноблочные операции,
а также на реальной нейронной сети BERT, содержащей 478 операторов. \\[1 cm]

\newpage

\begin{center}
    \textbf{Abstract} \\[0.3 cm]
    Разработка компилятора нейронных сетей на основе инфраструктуры MLIR для процессора с матричной архитектурой \\
    \textit{Вязовцев Андрей Викторович} \\[1 cm]
\end{center}

Neural networks are a good and effective way to process any data.
For this reason, they are used in various areas of life and are becoming
increasingly popular. To increase the performance of neural networks, various
methods are used, such as compilers and special computing devices.

This paper examines various methods for optimizing neural networks and describes
the implementation of the compiler. The goal of the work is to develop a neural
network compiler for Ascend processors based on the DaVinci architecture using
the LLVM MLIR infrastructure and ensure the generation of efficient machine code
in it. In particular, individual large-block operations of neural networks and
the process of their translation into instructions of the target architecture
are considered.

\newpage