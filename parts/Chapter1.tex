\section{Постановка задачи}
\label{sec:Chapter1} \index{Chapter1}

Задача исследования: разработать компилятор нейронных сетей для процессоров
Ascend, основанных на архитектруре DaVinci, с использованием инфраструктуры
LLVM MLIR и обеспечить генерацию эффективного машинного кода в нём.
Для достижения данной задачи были поставлены следующие цели исследования:

\begin{enumerate}
    \item Исследовать архитектуру современных популярных нейронных сетей
          и типичные для них операции.
    \item Исследовать подходы к эффективному исполнению нейронных сетей,
          в том числе использование специальных процессоров (NPU),
          компиляторов с разными целевыми архитектурами (CPU, GPU, NPU),
          узнать их особенности и используемые в них оптимизации.
    \item Изучить инфраструктуру LLVM MLIR и предоставляемые ею возможности
          для написания собственного компилятора.
    \item Исследовать архитектуру DaVinci, принцип работы нейроматричного
          процессора и её язык ассемблера.
    \item Разработать набор операторов для целевой архитектуры DaVinci в
          инфраструктуре MLIR.
    \item Исследовать и предложить методы генерации оптимального машинного
          кода для некоторых типичных операций нейронных сетей.
    \item Реализовать наиболее эффективные способы генерации машинного кода,
          исследовать их производительность.
\end{enumerate}

\newpage
