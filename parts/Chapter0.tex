\section{Введение}
\label{sec:Chapter0} \index{Chapter0}

Нейронные сети в последниее время испытывают большой подъём.
Это происходит, прежде всего, благодаря успехам Chat GPT,
которая показала новые возможность для обработки естественной речи.
Стоит отметить, что развитие этой сферы происходит не только за счёт
совершенствования точности ответов нейронных сетей. Например,
разбатываются процессоры с матричной архитектурой, которые могут
быть встроены в смартфоны. Очевидно, что такие решения будут
востребованы на мобильном рынке, который занимает крупную часть
всего IT-рынка.

Для использования таких процессоров необходим широкий набор утилит,
в том числе и компиляторы. Основной задачей любого компилятора
является получение наиболее оптимального с точки зрения
производительности машинного кода при сохранении всех свойств
исходной программы. Заметим, что в таких компиляторах помимо
традиционных техник оптимизации, таких как удаление мёртвого кода,
распостранения констант, сокращения общих подвыражений и других,
должны применяться другие техники, связанные с математическими
свойствами тензоров и спецификой целевой архитектуры.

В силу описанных выше причин идут активные исследования в
области компиляторов для нейронных сетей, в том числе и нашей
лабораторией. В данной работе будет исследованы особенности
целевой архитектуры и представлены способы генерации эффективого
машинного кода. 

\newpage
